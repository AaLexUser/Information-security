% !TEX program = xelatex
\documentclass[a4paper, 14pt]{report}

% Подключение необходимых пакетов для XeLaTeX
\usepackage{fontspec}                 % Для управления шрифтами
\usepackage{polyglossia}              % Для поддержки многоязычности
\setmainlanguage{russian}             % Установка основного языка
\setotherlanguage{english}            % Дополнительный язык (если требуется)
\newfontfamily\cyrillicfonttt{Courier New} % Шрифт для моноширинного текста
% Установка основного шрифта
\setmainfont{Times New Roman}

% Пакеты для оформления документа
\usepackage{geometry}                 % Параметры полей
\usepackage{setspace}                 % Межстрочный интервал
\usepackage{titlesec}                 % Настройка заголовков
\usepackage{graphicx}                 % Вставка изображений
\usepackage{caption}                  % Настройка подписей
\usepackage{tocloft}                  % Настройка оглавления
\usepackage{hyperref}                 % Гиперссылки
\usepackage{listings}                 % Вставка исходного кода
\usepackage{xcolor}                   % Цвета для оформления кода

% Настройка геометрии страницы
\geometry{
    a4paper,
    left=30mm,
    right=20mm,
    top=10mm,
    bottom=10mm,
    footskip=0mm
}


% Межстрочный интервал 1.5
\onehalfspacing

% Настройка заголовков
\titleformat{\chapter}[block]
  {\centering\bfseries\Large} % Форматирование заголовка
  {\chaptername\ \thechapter}{1em}{}

\titleformat{\section}
  {\centering\bfseries\large}
  {\thesection}{1em}{}

\titleformat{\subsection}
  {\centering\bfseries\normalsize}
  {\thesubsection}{1em}{}

% Оглавление без точек после номеров
\renewcommand{\cftsecleader}{\cftdotfill{\cftdotsep}}

% Подключение пакета listingsutf8 для поддержки UTF-8 в листингах
\usepackage{listingsutf8}

% Включение ссылок в оглавлении
\setcounter{tocdepth}{3}

% Настройки для листингов кода
\lstset{
    basicstyle=\ttfamily\small,
    keywordstyle=\color{blue},
    commentstyle=\color{gray},
    stringstyle=\color{red},
    numbers=left,
    numberstyle=\tiny,
    stepnumber=1,
    numbersep=5pt,
    tabsize=4,
    breaklines=true,
    breakatwhitespace=false,
    showstringspaces=false,
    frame=single
}
\makeatletter % see https://tex.stackexchange.com/a/320345
\lst@InputCatcodes
\def\lst@DefEC{%
 \lst@CCECUse \lst@ProcessLetter
  ^^80^^81^^82^^83^^84^^85^^86^^87^^88^^89^^8a^^8b^^8c^^8d^^8e^^8f%
  ^^90^^91^^92^^93^^94^^95^^96^^97^^98^^99^^9a^^9b^^9c^^9d^^9e^^9f%
  ^^a0^^a1^^a2^^a3^^a4^^a5^^a6^^a7^^a8^^a9^^aa^^ab^^ac^^ad^^ae^^af%
  ^^b0^^b1^^b2^^b3^^b4^^b5^^b6^^b7^^b8^^b9^^ba^^bb^^bc^^bd^^be^^bf%
  ^^c0^^c1^^c2^^c3^^c4^^c5^^c6^^c7^^c8^^c9^^ca^^cb^^cc^^cd^^ce^^cf%
  ^^d0^^d1^^d2^^d3^^d4^^d5^^d6^^d7^^d8^^d9^^da^^db^^dc^^dd^^de^^df%
  ^^e0^^e1^^e2^^e3^^e4^^e5^^e6^^e7^^e8^^e9^^ea^^eb^^ec^^ed^^ee^^ef%
  ^^f0^^f1^^f2^^f3^^f4^^f5^^f6^^f7^^f8^^f9^^fa^^fb^^fc^^fd^^fe^^ff%
  ^^^^20ac^^^^0153^^^^0152%
  % Basic Cyrillic alphabet coverage
  ^^^^0410^^^^0411^^^^0412^^^^0413^^^^0414^^^^0415^^^^0416^^^^0417%
  ^^^^0418^^^^0419^^^^041a^^^^041b^^^^041c^^^^041d^^^^041e^^^^041f%
  ^^^^0420^^^^0421^^^^0422^^^^0423^^^^0424^^^^0425^^^^0426^^^^0427%
  ^^^^0428^^^^0429^^^^042a^^^^042b^^^^042c^^^^042d^^^^042e^^^^042f%
  ^^^^0430^^^^0431^^^^0432^^^^0433^^^^0434^^^^0435^^^^0436^^^^0437%
  ^^^^0438^^^^0439^^^^043a^^^^043b^^^^043c^^^^043d^^^^043e^^^^043f%
  ^^^^0440^^^^0441^^^^0442^^^^0443^^^^0444^^^^0445^^^^0446^^^^0447%
  ^^^^0448^^^^0449^^^^044a^^^^044b^^^^044c^^^^044d^^^^044e^^^^044f%
  ^^^^0401^^^^0451%
  %%%
  ^^00}
\lst@RestoreCatcodes
\makeatother

\begin{document}


% Титульный лист
\begin{titlepage}
    \centering
    {\large Федеральное государственное автономное образовательное учреждение высшего образования}\\
    {\large «Национальный исследовательский университет ИТМО»}\\[0.5cm]

    {\large Факультет программной инженерии и компьютерной техники}\\[3cm]

    {\large \bfseries Лабораторная работа 2}\\[0.5cm]
    {\large \bfseries «Блочное симметричное шифрование»}\\[1cm]

    {\large Вариант № \underline{2в}}\\[5cm]
    \begin{flushright}
        {\large \underline{Группа: P34102}}\\[0.5cm]
        {\large \underline{Выполнил:} Лапин А.А.}\\[1cm]

        {\large \underline{Проверил:}}\\
        {\large Рыбаков С.Д.}\\[9cm]
    \end{flushright}

    {\large Санкт-Петербург}\\
    {\large 2024г.}
\end{titlepage}

\setcounter{page}{2}
% Оглавление
\tableofcontents
\newpage

% Введение
\chapter*{Введение}
\addcontentsline{toc}{chapter}{Введение}
Цель работы: изучение структуры и основных принципов работы современных алгоритмов блочного симметричного шифрования, приобретение навыков программной реализации блочных симметричных шифров.
% Раздел 1
\chapter{Текст задания}
Реализовать систему симметричного блочного шифрования, позволяющую шифровать и дешифровать файл на диске с использованием заданного блочного шифра в заданном режиме шифрования.

% Раздел 2: Структура проекта
\chapter{Структура проекта}
Структура проекта представлена следующим образом:

\begin{verbatim}
Cryptography/
└── lab1.2/
    ├── src/
    │   ├── idea.py
    │   ├── pcbc.py
    │   └── main.py
    ├── input.txt
    ├── encrypted.bin
    ├── decrypted.txt
    └── tests/
        └── test_idea_pcbc_cipher.py
\end{verbatim}

\section{Описание основных компонентов}

\subsection{idea.py}

Реализация алгоритма IDEA (International Data Encryption Algorithm). Предоставляет методы для выполнения основных операций шифрования и дешифрования блоков данных, а также управления ключевыми раундами.

\subsection{pcbc.py}
Реализация режима шифрования PCBC (Propagating Cipher Block Chaining). Обеспечивает безопасность шифрования за счет использования цепочки блоков и добавления пропагирующего эффекта ошибок при дешифровке.

\subsection{main.py}
Основной файл для шифрования и дешифровки файлов. Предоставляет интерфейс командной строки для выполнения операций шифрования и дешифровки, а также управления ключами и инициализационным вектором.

\subsection{test\_idea\_pcbc\_cipher.py}

Набор тестов для проверки корректности реализации алгоритма IDEA и режима шифрования PCBC. Использует библиотеку `pytest` для организации и выполнения тестовых случаев.


% Раздел 3: Листинги разработанной программы с комментариями
\chapter{Листинги разработанной программы с комментариями}
\section{idea.py}
\lstinputlisting[language=Python]{../src/idea.py}

\section{pcbc.py}
\lstinputlisting[language=Python]{../src/pcbc.py}

\section{main.py}
\lstinputlisting[language=Python]{../src/main.py}
% Раздел 4: Результаты работы программы
\chapter{Результаты работы программы}

\section{Пример 1: Шифрование}

\subsection{Исходный текст (input.txt)}
\begin{lstlisting}
Lorem Ipsum - это текст-"рыба", часто используемый в печати и вэб-дизайне. Lorem Ipsum является стандартной "рыбой" для текстов на латинице с начала XVI века. В то время некий безымянный печатник создал большую коллекцию размеров и форм шрифтов, используя Lorem Ipsum для распечатки образцов. Lorem Ipsum не только успешно пережил без заметных изменений пять веков, но и перешагнул в электронный дизайн. Его популяризации в новое время послужили публикация листов Letraset с образцами Lorem Ipsum в 60-х годах и, в более недавнее время, программы электронной вёрстки типа Aldus PageMaker, в шаблонах которых используется Lorem Ipsum.
\end{lstlisting}


\subsection{Команда для шифрования}
\begin{lstlisting}[language=bash]
python src/main.py encrypt input.txt encrypted.bin 16bytekeyforIDEA
\end{lstlisting}
\subsection{Результат}
Вывод в консоль:
\begin{lstlisting}
File encrypted and saved to encrypted.bin
\end{lstlisting}

\section{Пример 2: Дешифровка}

\subsection{Исходный текст (`encrypted.bin`):}
Двоичные данные, не представляемые в текстовом формате


\subsection{Команда для дешифровки}
\begin{lstlisting}[language=bash]
python src/main.py decrypt encrypted.bin decrypted.txt 16bytekeyforIDEA
\end{lstlisting}

\subsection{Результат}
Вывод в консоль:
\begin{lstlisting}
File decrypted and saved to decrypted.txt
\end{lstlisting}

\subsection{Файл decrypted.txt}
\begin{lstlisting}
Lorem Ipsum - это текст-"рыба", часто используемый в печати и вэб-дизайне. Lorem Ipsum является стандартной "рыбой" для текстов на латинице с начала XVI века. В то время некий безымянный печатник создал большую коллекцию размеров и форм шрифтов, используя Lorem Ipsum для распечатки образцов. Lorem Ipsum не только успешно пережил без заметных изменений пять веков, но и перешагнул в электронный дизайн. Его популяризации в новое время послужили публикация листов Letraset с образцами Lorem Ipsum в 60-х годах и, в более недавнее время, программы электронной вёрстки типа Aldus PageMaker, в шаблонах которых используется Lorem Ipsum.
\end{lstlisting}
\chapter{Заключение}
В ходе выполнения лабораторной работы была реализована система симметричного блочного шифрования на основе алгоритма IDEA в режиме PCBC. Программа успешно шифрует и дешифрует файлы, обеспечивая надежную защиту данных. Полученные результаты подтверждают корректность реализации алгоритма и режима шифрования.

\end{document}
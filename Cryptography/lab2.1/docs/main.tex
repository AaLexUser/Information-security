% !TEX program = xelatex
\documentclass[a4paper, 14pt]{report}

% Подключение необходимых пакетов для XeLaTeX
\usepackage{fontspec}                 % Для управления шрифтами
\usepackage{polyglossia}              % Для поддержки многоязычности
\setmainlanguage{russian}             % Установка основного языка
\setotherlanguage{english}            % Дополнительный язык (если требуется)
\newfontfamily\cyrillicfonttt{Courier New} % Шрифт для моноширинного текста
% Установка основного шрифта
\setmainfont{Times New Roman}

% Пакеты для оформления документа
\usepackage{geometry}                 % Параметры полей
\usepackage{setspace}                 % Межстрочный интервал
\usepackage{titlesec}                 % Настройка заголовков
\usepackage{graphicx}                 % Вставка изображений
\usepackage{caption}                  % Настройка подписей
\usepackage{tocloft}                  % Настройка оглавления
\usepackage{hyperref}                 % Гиперссылки
\usepackage{listings}                 % Вставка исходного кода
\usepackage{xcolor}                   % Цвета для оформления кода
\usepackage{amsmath}
\usepackage{tcolorbox}
\usepackage{tabularx}

% Настройка геометрии страницы
\geometry{
    a4paper,
    left=30mm,
    right=20mm,
    top=10mm,
    bottom=10mm,
    footskip=0mm
}


% Межстрочный интервал 1.5
\onehalfspacing

% Настройка заголовков
\titleformat{\chapter}[block]
  {\centering\bfseries\Large} % Форматирование заголовка
  {\chaptername\ \thechapter}{1em}{}

\titleformat{\section}
  {\centering\bfseries\large}
  {\thesection}{1em}{}

\titleformat{\subsection}
  {\centering\bfseries\normalsize}
  {\thesubsection}{1em}{}

% Оглавление без точек после номеров
\renewcommand{\cftsecleader}{\cftdotfill{\cftdotsep}}

% Подключение пакета listingsutf8 для поддержки UTF-8 в листингах
\usepackage{listingsutf8}

% Включение ссылок в оглавлении
\setcounter{tocdepth}{3}

% Настройки для листингов кода
\lstset{
    basicstyle=\ttfamily\small,
    keywordstyle=\color{blue},
    commentstyle=\color{gray},
    stringstyle=\color{red},
    numbers=left,
    numberstyle=\tiny,
    stepnumber=1,
    numbersep=5pt,
    tabsize=4,
    breaklines=true,
    breakatwhitespace=false,
    showstringspaces=false,
    frame=single
}
\makeatletter % see https://tex.stackexchange.com/a/320345
\lst@InputCatcodes
\def\lst@DefEC{%
 \lst@CCECUse \lst@ProcessLetter
  ^^80^^81^^82^^83^^84^^85^^86^^87^^88^^89^^8a^^8b^^8c^^8d^^8e^^8f%
  ^^90^^91^^92^^93^^94^^95^^96^^97^^98^^99^^9a^^9b^^9c^^9d^^9e^^9f%
  ^^a0^^a1^^a2^^a3^^a4^^a5^^a6^^a7^^a8^^a9^^aa^^ab^^ac^^ad^^ae^^af%
  ^^b0^^b1^^b2^^b3^^b4^^b5^^b6^^b7^^b8^^b9^^ba^^bb^^bc^^bd^^be^^bf%
  ^^c0^^c1^^c2^^c3^^c4^^c5^^c6^^c7^^c8^^c9^^ca^^cb^^cc^^cd^^ce^^cf%
  ^^d0^^d1^^d2^^d3^^d4^^d5^^d6^^d7^^d8^^d9^^da^^db^^dc^^dd^^de^^df%
  ^^e0^^e1^^e2^^e3^^e4^^e5^^e6^^e7^^e8^^e9^^ea^^eb^^ec^^ed^^ee^^ef%
  ^^f0^^f1^^f2^^f3^^f4^^f5^^f6^^f7^^f8^^f9^^fa^^fb^^fc^^fd^^fe^^ff%
  ^^^^20ac^^^^0153^^^^0152%
  % Basic Cyrillic alphabet coverage
  ^^^^0410^^^^0411^^^^0412^^^^0413^^^^0414^^^^0415^^^^0416^^^^0417%
  ^^^^0418^^^^0419^^^^041a^^^^041b^^^^041c^^^^041d^^^^041e^^^^041f%
  ^^^^0420^^^^0421^^^^0422^^^^0423^^^^0424^^^^0425^^^^0426^^^^0427%
  ^^^^0428^^^^0429^^^^042a^^^^042b^^^^042c^^^^042d^^^^042e^^^^042f%
  ^^^^0430^^^^0431^^^^0432^^^^0433^^^^0434^^^^0435^^^^0436^^^^0437%
  ^^^^0438^^^^0439^^^^043a^^^^043b^^^^043c^^^^043d^^^^043e^^^^043f%
  ^^^^0440^^^^0441^^^^0442^^^^0443^^^^0444^^^^0445^^^^0446^^^^0447%
  ^^^^0448^^^^0449^^^^044a^^^^044b^^^^044c^^^^044d^^^^044e^^^^044f%
  ^^^^0401^^^^0451%
  %%%
  ^^00}
\lst@RestoreCatcodes
\makeatother

\newcommand{\chtb}[3]{
  \begin{center}
    \begin{tabularx}{#1}{X|X}
    $\begin{aligned}
          #2
      \end{aligned}$ &
    \begin{minipage}{1em}
      $\Rightarrow$
    \end{minipage}
    $\begin{aligned}
         #3
      \end{aligned}$
    \end{tabularx}
  \end{center}
}

\lstdefinelanguage{YAML}{
  keywords={true,false,null,y,n},
  sensitive=true,
  comment=[l]{\#},
  morecomment=[s]{/*}{*/},
  stringstyle=\color{red},
  basicstyle=\ttfamily\small
}

\begin{document}


% Титульный лист
\begin{titlepage}
    \centering
    {\large Федеральное государственное автономное образовательное учреждение высшего образования}\\
    {\large «Национальный исследовательский университет ИТМО»}\\[0.5cm]

    {\large Факультет программной инженерии и компьютерной техники}\\[3cm]

    {\large \bfseries Лабораторная работа 3}\\[0.5cm]
    {\large \bfseries «Атака на алгоритм шифрования RSA посредством метода Ферма»}\\[1cm]

    {\large Вариант № \underline{4}}\\[5cm]
    \begin{flushright}
        {\large \underline{Группа: P34102}}\\[0.5cm]
        {\large \underline{Выполнил:} Лапин А.А.}\\[1cm]

        {\large \underline{Проверил:}}\\
        {\large Рыбаков С.Д.}\\[9cm]
    \end{flushright}

    {\large Санкт-Петербург}\\
    {\large 2024г.}
\end{titlepage}

\setcounter{page}{2}
% Оглавление
\tableofcontents
\newpage

% Введение
\chapter*{Введение}\phantomsection
\addcontentsline{toc}{chapter}{Введение}
Цель работы: изучить атаку на алгоритм шифрования RSA посредством метода Ферма.
% Раздел 1
\section*{Текст задания}
\begin{table}[h!]
    \centering
    \begin{tabular}{|l|l|l|l|}
        \hline
        Вариант        &
        Модуль, N      &
        Экспонента, е  &
        Блок зашифрованного текста, C                                                                                                                                                                                                                 \\ \hline
        4              &
        89318473363897 &
        2227661        &
        \begin{tabular}[c]{@{}l@{}}3403106899606\\ 26746900101177\\ 67769260919924\\ 77873792354218\\ 15782947730235\\ 15100267747684\\ 28877721728826\\ 62898555111378\\ 4989704651236\\ 55293402838380\\ 4108112294245\\ 8492269964172\end{tabular} \\ \hline
    \end{tabular}
\end{table}
% Раздел 2: Структура проекта

\chapter*{Ход работы}\phantomsection
\addcontentsline{toc}{chapter}{Ход работы}
\section*{Теория:}\phantomsection
\addcontentsline{toc}{section}{Теория}

\begin{tcolorbox}[colback=white!95!gray, colframe=black, title=Метод факторизации Ферма]
    Если $N > 0$ и $N$ нечетное, то существует взаимно однозначное соответствие между разложением на множители
    $n = (x - y) \cdot (x + y)$ и представлением в виде разности квадратов $n = x^2 - y^2$ с $x > y > 0$.
\end{tcolorbox}

\begin{center}
    \begin{tabularx}{\textwidth}{p{4em}|p{10em}|p{8em}|p{15em}}
        $\begin{aligned}
                  & p = x - y \\
                  & q = x + y \\
             \end{aligned}$          &
        \begin{minipage}{1em}
            $\Rightarrow$
        \end{minipage}
        $\begin{aligned}
                  & x = p + q = 2x  \\
                  & y = p - q = -2y \\
             \end{aligned}$     &
        \begin{minipage}{1em}
            $\Rightarrow$
        \end{minipage}
        $\begin{aligned}
                  & x = \frac{p + q}{2} \\
                  & y = \frac{q - p}{2}
             \end{aligned}$ &
        \begin{minipage}{1em}
            $\Rightarrow$
        \end{minipage}
        $\begin{aligned}
                  & N = \left( \frac{p + q}{2} \right)^2 - \left( \frac{q - p}{2} \right)^2 \\
             \end{aligned}$
    \end{tabularx}
\end{center}

Если $p$ и $q$ близки друг к другу, то $\left( \frac{q - p}{2} \right)^2 \to 0$, и $N \approx \left( \frac{p + q}{2} \right)^2$.

Пусть $t = \frac{p + q}{2}$, а $s = \frac{q - p}{2}$, тогда $N = t^2 - s^2$.

Тогда $t^2 - N = s^2$.

$t \approx \sqrt{N}$.

Найдем $t$ методом перебора, начиная с $\lceil \sqrt{N} \rceil$.

В результате вычисления $t^2 - N$ мы должны получить квадрат некоторого целого числа $s$.

$p = t + s$ и $q = t - s$.

$\phi(N) = (p - 1) \cdot (q - 1)$.

$d = e^{-1} \mod \phi(N)$.

\section*{Вычисление руками:}\phantomsection
\addcontentsline{toc}{section}{Вычисление руками}

$t = \lceil \sqrt{N} \rceil = 9450846$

$t^2 - N = 9450846^2 - 89318473363897 = 16751819$

$s = \sqrt{16751819} = 4092.89$\\[0.5cm]


$t = 9450846 + 1 = 9450847$

$t^2 - N = 9450847^2 - 89318473363897 = 35653512$

$s = \sqrt{35653512} = 5971.05$\\[0.5cm]



$t = 9450847 + 1 = 9450848$

$t^2 - N = 9450848^2 - 89318473363897 = 54555207$

$s = \sqrt{54555207} = 7386.14$\\[0.5cm]



$t = 9450848 + 1 = 9450849$

$t^2 - N = 9450849^2 - 89318473363897 = 73456904$

$s = \sqrt{73456904} = 8570.7$\\[0.5cm]


$t = 9450849 + 1 = 9450850$

$t^2 - N = 9450850^2 - 89318473363897 = 92358603$

$s = \sqrt{92358603} = 9610.3$\\[0.5cm]


$t = 9450850 + 1 = 9450851$

$t^2 - N = 9450851^2 - 89318473363897 = 111260304$

$s = \sqrt{111260304} = 10548.0$\\[0.5cm]


$p = t + s = 9450851 + 10548 = 9461399$

$q = t - s = 9450851 - 10548 = 9440303$

$\phi(N) = (p - 1) \cdot (q - 1) = 9461398 \cdot 9440302 = 89318454462196$

$d = e^{-1} \mod \phi(N) = 2227661^{-1} \mod 89318454462196 = 15910526683025$

\section*{Решение на Python:}\phantomsection
\addcontentsline{toc}{section}{Решение на Python}
\lstinputlisting[language=Python, caption=main.py]{../main.py}
\lstinputlisting[language=YAML, caption=config.yaml]{../config.yaml}

% Раздел 4: Результаты работы программы
\section*{Результаты работы программы}\phantomsection
\addcontentsline{toc}{section}{Результаты работы программы}

\begin{lstlisting}[language=YAML, caption=Вывод в консоль]
> python main.py
N = 89318473363897
e = 2227661
Ciphertexts = [3403106899606, 26746900101177, 67769260919924, 77873792354218, 15782947730235, 15100267747684, 28877721728826, 62898555111378, 4989704651236, 55293402838380, 4108112294245, 8492269964172]
Factoring N using Fermat's method...
t = 9450851, s = 10548
Factors found: p = 9461399, q = 9440303
phi(N) = 89318454462196
Private exponent d = 15910526683025
Decrypting ciphertext blocks...
Plaintext: одномаршрутный (single route) и всемаршрутный (a
\end{lstlisting}
\chapter*{Заключение}\phantomsection
\addcontentsline{toc}{chapter}{Заключение}
В ходе выполнения лабораторной работы была реализована атака на алгоритм шифрования RSA посредством метода Ферма.


\end{document}
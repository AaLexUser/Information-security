% !TEX program = xelatex
\documentclass[a4paper, 14pt]{report}

% Подключение необходимых пакетов для XeLaTeX
\usepackage{fontspec}                 % Для управления шрифтами
\usepackage{polyglossia}              % Для поддержки многоязычности
\setmainlanguage{russian}             % Установка основного языка
\setotherlanguage{english}            % Дополнительный язык (если требуется)
\newfontfamily\cyrillicfonttt{Courier New} % Шрифт для моноширинного текста
% Установка основного шрифта
\setmainfont{Times New Roman}

% Пакеты для оформления документа
\usepackage{geometry}                 % Параметры полей
\usepackage{setspace}                 % Межстрочный интервал
\usepackage{titlesec}                 % Настройка заголовков
\usepackage{graphicx}                 % Вставка изображений
\usepackage{caption}                  % Настройка подписей
\usepackage{tocloft}                  % Настройка оглавления
\usepackage{hyperref}                 % Гиперссылки
\usepackage{listings}                 % Вставка исходного кода
\usepackage{xcolor}                   % Цвета для оформления кода
\usepackage{amsmath}
\usepackage{tcolorbox}
\usepackage{tabularx}
\usepackage{multirow}

% Настройка геометрии страницы
\geometry{
    a4paper,
    left=30mm,
    right=20mm,
    top=10mm,
    bottom=10mm,
    footskip=0mm
}


% Межстрочный интервал 1.5
\onehalfspacing

% Настройка заголовков
\titleformat{\chapter}[block]
  {\centering\bfseries\Large} % Форматирование заголовка
  {\chaptername\ \thechapter}{1em}{}

\titleformat{\section}
  {\centering\bfseries\large}
  {\thesection}{1em}{}

\titleformat{\subsection}
  {\centering\bfseries\normalsize}
  {\thesubsection}{1em}{}

% Оглавление без точек после номеров
\renewcommand{\cftsecleader}{\cftdotfill{\cftdotsep}}

% Подключение пакета listingsutf8 для поддержки UTF-8 в листингах
\usepackage{listingsutf8}

% Включение ссылок в оглавлении
\setcounter{tocdepth}{3}

% Настройки для листингов кода
\lstset{
    basicstyle=\ttfamily\small,
    keywordstyle=\color{blue},
    commentstyle=\color{gray},
    stringstyle=\color{red},
    numbers=left,
    numberstyle=\tiny,
    stepnumber=1,
    numbersep=5pt,
    tabsize=4,
    breaklines=true,
    breakatwhitespace=false,
    showstringspaces=false,
    frame=single
}
\makeatletter % see https://tex.stackexchange.com/a/320345
\lst@InputCatcodes
\def\lst@DefEC{%
 \lst@CCECUse \lst@ProcessLetter
  ^^80^^81^^82^^83^^84^^85^^86^^87^^88^^89^^8a^^8b^^8c^^8d^^8e^^8f%
  ^^90^^91^^92^^93^^94^^95^^96^^97^^98^^99^^9a^^9b^^9c^^9d^^9e^^9f%
  ^^a0^^a1^^a2^^a3^^a4^^a5^^a6^^a7^^a8^^a9^^aa^^ab^^ac^^ad^^ae^^af%
  ^^b0^^b1^^b2^^b3^^b4^^b5^^b6^^b7^^b8^^b9^^ba^^bb^^bc^^bd^^be^^bf%
  ^^c0^^c1^^c2^^c3^^c4^^c5^^c6^^c7^^c8^^c9^^ca^^cb^^cc^^cd^^ce^^cf%
  ^^d0^^d1^^d2^^d3^^d4^^d5^^d6^^d7^^d8^^d9^^da^^db^^dc^^dd^^de^^df%
  ^^e0^^e1^^e2^^e3^^e4^^e5^^e6^^e7^^e8^^e9^^ea^^eb^^ec^^ed^^ee^^ef%
  ^^f0^^f1^^f2^^f3^^f4^^f5^^f6^^f7^^f8^^f9^^fa^^fb^^fc^^fd^^fe^^ff%
  ^^^^20ac^^^^0153^^^^0152%
  % Basic Cyrillic alphabet coverage
  ^^^^0410^^^^0411^^^^0412^^^^0413^^^^0414^^^^0415^^^^0416^^^^0417%
  ^^^^0418^^^^0419^^^^041a^^^^041b^^^^041c^^^^041d^^^^041e^^^^041f%
  ^^^^0420^^^^0421^^^^0422^^^^0423^^^^0424^^^^0425^^^^0426^^^^0427%
  ^^^^0428^^^^0429^^^^042a^^^^042b^^^^042c^^^^042d^^^^042e^^^^042f%
  ^^^^0430^^^^0431^^^^0432^^^^0433^^^^0434^^^^0435^^^^0436^^^^0437%
  ^^^^0438^^^^0439^^^^043a^^^^043b^^^^043c^^^^043d^^^^043e^^^^043f%
  ^^^^0440^^^^0441^^^^0442^^^^0443^^^^0444^^^^0445^^^^0446^^^^0447%
  ^^^^0448^^^^0449^^^^044a^^^^044b^^^^044c^^^^044d^^^^044e^^^^044f%
  ^^^^0401^^^^0451%
  %%%
  ^^00}
\lst@RestoreCatcodes
\makeatother

\newcommand{\chtb}[3]{
  \begin{center}
    \begin{tabularx}{#1}{X|X}
    $\begin{aligned}
          #2
      \end{aligned}$ &
    \begin{minipage}{1em}
      $\Rightarrow$
    \end{minipage}
    $\begin{aligned}
         #3
      \end{aligned}$
    \end{tabularx}
  \end{center}
}

\lstdefinelanguage{YAML}{
  keywords={true,false,null,y,n},
  sensitive=true,
  comment=[l]{\#},
  morecomment=[s]{/*}{*/},
  stringstyle=\color{red},
  basicstyle=\ttfamily\small
}

\begin{document}


% Титульный лист
\begin{titlepage}
    \centering
    {\large Федеральное государственное автономное образовательное учреждение высшего образования}\\
    {\large «Национальный исследовательский университет ИТМО»}\\[0.5cm]

    {\large Факультет программной инженерии и компьютерной техники}\\[3cm]

    {\large \bfseries Лабораторная работа 5}\\[0.5cm]
    {\large \bfseries «Атака на алгоритм шифрования RSA методом бесключевого
    чтения»}\\[1cm]

    {\large Вариант № \underline{12}}\\[5cm]
    \begin{flushright}
        {\large \underline{Группа: P34102}}\\[0.5cm]
        {\large \underline{Выполнил:} Лапин А.А.}\\[1cm]

        {\large \underline{Проверил:}}\\
        {\large Рыбаков С.Д.}\\[9cm]
    \end{flushright}

    {\large Санкт-Петербург}\\
    {\large 2024г.}
\end{titlepage}

\setcounter{page}{2}
% Оглавление
\tableofcontents
\newpage

% Введение
\chapter*{Введение}\phantomsection
\addcontentsline{toc}{chapter}{Введение}
Цель работы: изучить атаку на алгоритм шифрования RSA посредством метода бесключевого чтения.
% Раздел 1
\section*{Текст задания}
\begin{table}[h]
    \centering
    \begin{tabular}{|l|l|ll|ll|}
        \hline
        \multirow{2}{*}{Вариант}                                                                                                                                         &
        \multirow{2}{*}{Модуль, N}                                                                                                                                       &
        \multicolumn{2}{l|}{Экспоненты}                                                                                                                                  &
        \multicolumn{2}{l|}{Блок зашифрованного текста}                                                                                                                    \\ \cline{3-6}
                                                                                                                                                                         &
                                                                                                                                                                         &
        \multicolumn{1}{l|}{e1}                                                                                                                                          &
        e2                                                                                                                                                               &
        \multicolumn{1}{l|}{C1}                                                                                                                                          &
        C2                                                                                                                                                                 \\ \hline
        12                                                                                                                                                               &
        385751370271                                                                                                                                                     &
        \multicolumn{1}{l|}{365797}                                                                                                                                      &
        1109663                                                                                                                                                          &
        \multicolumn{1}{l|}{\begin{tabular}[c]{@{}l@{}}58541562205\\ 167003685579\\ 381877628242\\ 256218527098\\ 164244249864\\ 6588741823\\ 180308234660\end{tabular}} &
        \begin{tabular}[c]{@{}l@{}}78032032470\\ 13064174635\\ 326727914830\\ 364066420370\\ 177576861402\\ 65863828523\\ 111437045566\end{tabular}                        \\ \hline
    \end{tabular}
\end{table}
% Раздел 2: Структура проекта

\chapter*{Ход работы}\phantomsection
\addcontentsline{toc}{chapter}{Ход работы}
Будем строить последовательность:
$c_1 = c, ~c_i = c_{i-1}^{e} \mod N, i > 1$.
\begin{center}
    \begin{tabularx}{\textwidth}{p{0.2\textwidth}|X}
        $\begin{aligned}
                  & N = 385751370271     \\
                  & e_1 = 365797         \\
                  & e_2 = 1109663        \\
                  & c_1 = 58541562205... \\
                  & c_2 = 78032032470... \\
             \end{aligned}$ &
        \begin{minipage}{1em}
            $\Rightarrow$
        \end{minipage}
        $\begin{aligned}
                  & c_1 = m^{e_1} \mod N                                                                                         \\
                  & c_2 = m^{e_2} \mod N                                                                                         \\
                  & \text{Используя расширенный алгоритм Евклида, находим } r \text{ и } s \text{, такие что } r e_1 + s e_2 = 1 \\
                  & \text{Тогда } m = c_1^r c_2^s \mod N
             \end{aligned}$
    \end{tabularx}
\end{center}

\section*{Программная реализация}\phantomsection
\addcontentsline{toc}{section}{Программная реализация}
\lstinputlisting[language=Python, caption=main.py]{../main.py}
\lstinputlisting[language=YAML, caption=config.yaml]{../config.yaml}



% Раздел 4: Результаты работы программы
\section*{Результаты работы программы}\phantomsection
\addcontentsline{toc}{section}{Результаты работы программы}

\begin{lstlisting}[language=YAML, caption=Вывод в консоль]
> python main.py
N = 385751370271
e1 = 365797
e2 = 1109663
c1 = [58541562205, 167003685579, 381877628242, 256218527098, 164244249864, 6588741823, 180308234660, 174572441677, 259951955034, 378589342820, 319378579620, 21405495597, 226860843155]
c2 = [78032032470, 13064174635, 326727914830, 364066420370, 177576861402, 65863828523, 111437045566, 124743274954, 119577259869, 85769669875, 4688914942, 261002397567, 341722428571]
Performing Infinite Reading Attack on ciphertext pairs...
Decrypting Ciphertext Pairs: 100%|████████████████████████████████| 13/13 [00:00<00:00, 53667.28it/s]
Plaintext: из исходного пакета, в котором переданы эти 900 б.
\end{lstlisting}
\chapter*{Заключение}\phantomsection
\addcontentsline{toc}{chapter}{Заключение}
В ходе выполнения лабораторной работы была реализована атака на алгоритм шифрования RSA методом бесключевого чтения.
\end{document}